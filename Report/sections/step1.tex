\subsection{Description of the objective}

In the first step, only three entities are created: Alice, Bob and Charlie. The two peers need to read configuration files to know their own IP addresses. The tracker is not implemented yet, the client has to read a configuration file in order to know where the chunks are located.

Knowing the message format used in the project\footnote{\url{https://github.com/jpagex/elec-h-417-project/blob/master/statement.pdf}}
,the peers will analyze the requests they receive from the client and they should be able to either send the chunk that was asked or to return an error message.

\subsection{Proposed solution}

Using object-oriented Python programming, classes \texttt{peers} and \texttt{client} have been created. Alice and Bob are thus \texttt{peers} objects and Charlie is a \texttt{client} object.

All the data transfer between the peers and the clients are handled using TCP to ensure that every requested chunk is correctly transmitted (provided a peer owns it).

\subsubsection{Peers}

The way peers are implemented is quite straightforward. When instantiated, the peer opens a socket and stays idle while waiting for a connection. When a connection is received from the client, a single thread by peer is launched.

This thread will analyze the incoming messages and will send back the appropriate errors when necessary. If the message format and content is correct, the peer will send the requested chunk to the client.

The thread will run and send errors through the socket while there are incoming messages from the client. If the client stops communicating, the peer will assume that all the requests have been sent and it will close the socket and the connection.

\subsubsection{Client}

The client are implemented in order to achieve the maximum file transfer speed, while handling cases where the configuration file declaring which peer owns which chunk is not correct.

Two threads are created so that the chunks are requested and downloaded from Alice and Bob in a parallel way. The library \texttt{Queue} is used, and three queues are created when the client is instantiated. The first and second queues contain the list of chunks that are only owned by one of the two peers. In this way, each thread is sending requests to a different peer and it is only asking for chunks that are not shared. When the queue associated to a thread is empty, it will request chunks that are owned by the two peers, which are listed in the third queue. This implementation allows to ensure that if one the connections is slower that the other, the impact on the file transfer speed will be as little as possible.


\subsection{Sequence diagram}